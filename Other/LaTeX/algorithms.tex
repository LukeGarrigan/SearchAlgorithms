%%%%%%%%%%%%%%%%%%%%%%%%%%%%%%%%%%%%%%%%%%%%%%%%%%%%%%%%
%
% Change the option between square brackets
% depending on the document you have to write:
%
% proposal    for the initial proposal
% review      for the literature review
% progress    for the progress report
% final       for the final report
% 
%%%%%%%%%%%%%%%%%%%%%%%%%%%%%%%%%%%%%%%%%%%%%%%%%%%%%%%%

\documentclass[progress]{cmpreport}
\makeatletter
\input{t1pcr.fd}
\makeatother
\setlength{\footnotesep}{3ex}
\usepackage{listings}
\usepackage{pgfplots}
% Some package I am using. You may not need them
%
\usepackage{rotating}
\usepackage{subfloat}

% packages for algorithms
\usepackage{amsmath}
\usepackage{algorithm}
\usepackage[noend]{algpseudocode}

\usepackage{amsmath}
\usepackage{algorithm}
\usepackage[noend]{algpseudocode}



%\setkeys{Gin}{draft}

%%%%%%%%%%%%%%%%%%%%%%%%%%%%%%%%%%%%%%%%%%%%%%%%%%%%%%%%
%
%  Fill in the fields with:
%
%  your project title
%  your name
%  your registration number
%  your supervisor's name
%
%%%%%%%%%%%%%%%%%%%%%%%%%%%%%%%%%%%%%%%%%%%%%%%%%%%%%%%%
\title{File for the algorithms}

%%%%%%%%%%%%%%%%%%%%%%%%%%%%%%%%%%%%%%%%%%%%%%%%%%%%%%%%
%
% The author's name is ignored if the following command 
% is not present in the document
%
% Before submitting a PDF of your final report to the 
% project database you may comment out the command
% if you are worried about lack of anonimity.
%
%%%%%%%%%%%%%%%%%%%%%%%%%%%%%%%%%%%%%%%%%%%%%%%%%%%%%%%%
\author{Luke Garrigan}


\registration{100086495}
\supervisor{Dr Pierre Chardaire}
%%%%%%%%%%%%%%%%%%%%%%%%%%%%%%%%%%%%%%%%%%%%%%%%%%%%%%%%
%
% Fill in the field with your module code.
% this should be:
%
% for BIS            -> CMP-6012Y
% for BUSINESS STATS -> CMP-6028Y
% for other students -> CMP-6013Y
%
%%%%%%%%%%%%%%%%%%%%%%%%%%%%%%%%%%%%%%%%%%%%%%%%%%%%%%%%
\ccode{CMP-6012Y}


\summary{
}

\acknowledgements{

}

%%%%%%%%%%%%%%%%%%%%%%%%%%%%%%%%%%%%%%%%%%%%%%%%%%%%%%%%%%%%%%%%%%
%
% If you do not want a list of figures and a list of tables
% to appear after the table of content then uncomment this line 
%
% Note that the class file contains code to avoid
% producing an empty list section (e.g list of figures) if the 
% list is empty (i.e. no figure in document).
%
% The command also prevents inserting a list of figures or tables 
% anywhere else in the document
%
% Some supervisors think that a report should not contain these
% lists. Please ask your supervisor's opinion.
%
%%%%%%%%%%%%%%%%%%%%%%%%%%%%%%%%%%%%%%%%%%%%%%%%%%%%%%%%%%%%%%%%%%
%\nolist,

%%%%%%%%%%%%%%%%%%%%%%%%%%%%%%%%%%%%%%%%%%%%%%%%%%%%%%%%%%%%%%%%%%
%
% Comment out if you want your list of figures and list of
% tables on two or more pages, in particular if the lists do not fit 
% on a single page.
%
%%%%%%%%%%%%%%%%%%%%%%%%%%%%%%%%%%%%%%%%%%%%%%%%%%%%%%%%%%%%%%%%%%
\onePageLists

\begin{document}
	
	
	\section{Introduction}

	

	\makeatletter
	\def\BState{\State\hskip-\ALG@thistlm}
	\makeatother
	%this changes the style of the comments 
	%\algrenewcommand{\algorithmiccomment}[1]{\hskip3em$\rightarrow$ #1}
	
	\begin{algorithm}
		\caption{Manhattan Distance}\label{Manhattan Distance}
		\begin{algorithmic}[1]
			\Procedure{ManDist}{$state$}	\Comment{The current puzzle configuration }
			\State$total\gets 0$
			\State$puzzleLength\gets state.size()$
			\State$dimensions\gets \sqrt{puzzleLength}$
			\For{$i\gets 1, puzzleLength$}	\Comment{Loops through each tile of the puzzle}
				\State $tileValue\gets state[i]$
				\State $expectedRow\gets (tileValue -1)\div dimensions$
				\State $expectedCol\gets (tileValue -1)\mod dimensions$
				\State $rowNum \gets i\div dimensions$
				\State $rowNum\gets i \bmod dimensions$
				\State $total\gets total \texttt{+} \mid{expectedRow-rowNum}\mid\texttt{+}\mid{expectedCol-colNum}\mid$
			\EndFor
			\State \textbf{return} $total$\Comment{The heuristic is the total}
			\EndProcedure
		\end{algorithmic}
	\end{algorithm}

	
	
	
	\begin{algorithm}
		\caption{Iterative Deepening A Star}\label{IDAStar}
		\begin{algorithmic}[1]
			
			\State $state$ \Comment The current puzzle configuration
			\State $g(state)$ \Comment The cost to reach the current state
			\State $h(state)$ \Comment Estimated cost of the cheapest path from state to goal
			\State $f(state) \gets g(state)+h(state)$
			\State $neighbours(state)$ \Comment Expands possible moves from current state ordered by g + h
			\Statex
			\Procedure{IDAStar}{$state$}
				\State $bound \gets f(state)$
				\While{ \textbf{ not } solved}\Comment{Loops until a solution is found}
					\State $ bound \gets DFS(state, bound)$	\Comment Performs a bounded depth-first search 
				\EndWhile\label{}
			\EndProcedure
			\Statex
			
			\Procedure{dfs}{$state, bound$}
		    \If{$f(state) > bound$}
				\State \textbf{return} $f(state)$
			\EndIf	
			
			\If{$h(state) == 0$} \Comment No more moves needed to reach goal state
				\State \textbf{return} $solved$
			\EndIf
			\State $min \gets \infty$	
			\For{$neighbour \textbf{ in } neighbours(state)$}	
				\State $temp \gets DFS(neighbour, bound)$
				
				\If{$temp < min}$
					\State $min \gets temp $
				\EndIf
			\EndFor
			\State \textbf{return} $min$\Comment{Returns the smallest of the neighbours}
			\EndProcedure
			
		\end{algorithmic}
    \end{algorithm}




	
		
	\begin{algorithm}
		\caption{Breadth-First Search}\label{BFS}
		\begin{algorithmic}[1]
			\Procedure{BFS}{$state$}
			\State $s \gets \textit{empty set}$
			\State $q \gets \textit{empty queue}$
			\State $q.add(state)$
			\State $q.enqueue(state)$
			\While{$q.size() > 0$}
				\State $currentState \gets q.dequeue()$
				\If {$currentState == goal$}
					\State \textbf{return} $current$
				\EndIf
				\For{$neighbour \textbf{ in } neighbours(currentState)$}	
					\If{$neighbour \text{ is not in } s$}
						\State $s.add(neighbour)$
						\State $q.enqueue(neighbour)$
					\EndIf
					
				\EndFor
			\EndWhile
			\EndProcedure
		\end{algorithmic}
	\end{algorithm}



	\begin{algorithm}
		\caption{Is Current State Solvable}\label{Solvable}
		\begin{algorithmic}[1]
				\Procedure{IsSolvable}{state}
					\State $stateLength \gets state.size()$
					\State 	$gridWidth \gets \sqrt{stateLength}$
					\State $row \gets 0$ \Comment The current row we are on
					\State $blankRow \gets 0$ \Comment The row with the blank tile
					
					\For{$i\gets 1, puzzleLength$}
						\If{$i \bmod gridWidth ==0 $}
							\State $row\texttt{++}$	
						\EndIf	
						\If{$state[i] == 0$}
							\State	$blankRow \gets row$
							\State \textbf{continue}
						\EndIf
						
						\For{$j\gets i \texttt{+} 1, puzzleLength$}
							\If{$state[i] > state[j] \textbf{ and } state[i] \not= 0$ }
								\State $parity\texttt{++}$	
							
							\EndIf
						\EndFor
					\EndFor
					
					\If{$gridWidth \bmod 2 == 0$}
						\If{$blankRow \bmod 2 ==0$}
							\State \textbf{return} $parity \bmod 2 == 0$
						\Else
							\State \textbf{return} $parity \bmod 2 \not= 0$
						\EndIf
						
					\Else 
						\State \textbf{return} $parity \bmod 2 == 0$
					\EndIf
				\EndProcedure
		\end{algorithmic}	
	\end{algorithm}	




	
	
	
\end{document}

