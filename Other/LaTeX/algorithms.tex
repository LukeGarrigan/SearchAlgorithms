%%%%%%%%%%%%%%%%%%%%%%%%%%%%%%%%%%%%%%%%%%%%%%%%%%%%%%%%
%
% Change the option between square brackets
% depending on the document you have to write:
%
% proposal    for the initial proposal
% review      for the literature review
% progress    for the progress report
% final       for the final report
% 
%%%%%%%%%%%%%%%%%%%%%%%%%%%%%%%%%%%%%%%%%%%%%%%%%%%%%%%%

\documentclass[progress]{cmpreport}
\makeatletter
\input{t1pcr.fd}
\makeatother
\setlength{\footnotesep}{3ex}
\usepackage{listings}
\usepackage{pgfplots}
% Some package I am using. You may not need them
%
\usepackage{rotating}
\usepackage{subfloat}

% packages for algorithms
\usepackage{amsmath}
\usepackage{algorithm}
\usepackage[noend]{algpseudocode}

\usepackage{amsmath}
\usepackage{algorithm}
\usepackage[noend]{algpseudocode}



%\setkeys{Gin}{draft}

%%%%%%%%%%%%%%%%%%%%%%%%%%%%%%%%%%%%%%%%%%%%%%%%%%%%%%%%
%
%  Fill in the fields with:
%
%  your project title
%  your name
%  your registration number
%  your supervisor's name
%
%%%%%%%%%%%%%%%%%%%%%%%%%%%%%%%%%%%%%%%%%%%%%%%%%%%%%%%%
\title{File for the algorithms}

%%%%%%%%%%%%%%%%%%%%%%%%%%%%%%%%%%%%%%%%%%%%%%%%%%%%%%%%
%
% The author's name is ignored if the following command 
% is not present in the document
%
% Before submitting a PDF of your final report to the 
% project database you may comment out the command
% if you are worried about lack of anonimity.
%
%%%%%%%%%%%%%%%%%%%%%%%%%%%%%%%%%%%%%%%%%%%%%%%%%%%%%%%%
\author{Luke Garrigan}


\registration{100086495}
\supervisor{Dr Pierre Chardaire}
%%%%%%%%%%%%%%%%%%%%%%%%%%%%%%%%%%%%%%%%%%%%%%%%%%%%%%%%
%
% Fill in the field with your module code.
% this should be:
%
% for BIS            -> CMP-6012Y
% for BUSINESS STATS -> CMP-6028Y
% for other students -> CMP-6013Y
%
%%%%%%%%%%%%%%%%%%%%%%%%%%%%%%%%%%%%%%%%%%%%%%%%%%%%%%%%
\ccode{CMP-6012Y}


\summary{
}

\acknowledgements{

}

%%%%%%%%%%%%%%%%%%%%%%%%%%%%%%%%%%%%%%%%%%%%%%%%%%%%%%%%%%%%%%%%%%
%
% If you do not want a list of figures and a list of tables
% to appear after the table of content then uncomment this line 
%
% Note that the class file contains code to avoid
% producing an empty list section (e.g list of figures) if the 
% list is empty (i.e. no figure in document).
%
% The command also prevents inserting a list of figures or tables 
% anywhere else in the document
%
% Some supervisors think that a report should not contain these
% lists. Please ask your supervisor's opinion.
%
%%%%%%%%%%%%%%%%%%%%%%%%%%%%%%%%%%%%%%%%%%%%%%%%%%%%%%%%%%%%%%%%%%
%\nolist,

%%%%%%%%%%%%%%%%%%%%%%%%%%%%%%%%%%%%%%%%%%%%%%%%%%%%%%%%%%%%%%%%%%
%
% Comment out if you want your list of figures and list of
% tables on two or more pages, in particular if the lists do not fit 
% on a single page.
%
%%%%%%%%%%%%%%%%%%%%%%%%%%%%%%%%%%%%%%%%%%%%%%%%%%%%%%%%%%%%%%%%%%
\onePageLists

\begin{document}
	
	
	\section{Introduction}

	

	\makeatletter
	\def\BState{\State\hskip-\ALG@thistlm}
	\makeatother
	%this changes the style of the comments 
	%\algrenewcommand{\algorithmiccomment}[1]{\hskip3em$\rightarrow$ #1}
	
	\begin{algorithm}
		\caption{Manhattan Distance}\label{Manhattan Distance}
		\begin{algorithmic}[1]
			\Procedure{ManDist}{$state$}	\Comment{The current puzzle configuration }
			\State$total\gets 0$
			\State$puzzleLength\gets sizeOf(state)$
			\State$dimensions\gets \sqrt{puzzleLength}$
			\For{$i\gets 1, puzzleLength$}	\Comment{Loops through each tile of the puzzle}
				\State $tileValue\gets state[i]$
				\State $expectedRow\gets \dfrac{(tileValue -1)}{dimensions}$
				\State $expectedCol\gets (tileValue -1)\bmod dimensions$
				\State $rowNum \gets \dfrac{i}{dimensions}$
				\State $rowNum\gets i \bmod dimensions$
				\State $total\gets total \texttt{+} \mid{expectedRow-rowNum}\mid\texttt{+}\mid{expectedCol-colNum}\mid$
			\EndFor
			\State \textbf{return} $total$\Comment{The heuristic is the total}
			\EndProcedure
		\end{algorithmic}
	\end{algorithm}

	
	
	
	\begin{algorithm}
		\caption{Iterative Deepening A Star}\label{IDAStar}
		\begin{algorithmic}[1]
			\Procedure{IDAStar}{$state$}
			\State$bound\gets currentHeuristic(state)$ 
			\EndProcedure
		\end{algorithmic}
    \end{algorithm}

	
	
\end{document}

